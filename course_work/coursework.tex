\documentclass[a4paper,fleqn,14pt,titlepage]{extarticle}
\usepackage[english,russian]{babel}
\usepackage[T2A]{fontenc}
\usepackage[utf8]{inputenc}

\usepackage{indentfirst} %отступ для первого абзаца
\usepackage{main_style} %подключение титульника
\linespread{1.1}



\usepackage{cmap} %чтобы можно было текст копировать

\usepackage{amssymb,amsfonts,amsmath,mathtext,enumerate,float}
\usepackage{libertine}
\usepackage{microtype}

\usepackage{tabularx} % работа с таблицами https://ru.overleaf.com/learn/latex/Tables

\usepackage{graphicx}
\usepackage{caption}
\graphicspath{{pictures}}
\usepackage{tikz}

% \RequirePackage[left=20mm,right=10mm,top=20mm,bottom=20mm,headsep=0pt,includefoot]{geometry}
% Если вас тошнит от поля в 10мм --- увеличивайте до 20-ти, ну и про переплёт не забывайте:
\usepackage{geometry}
\geometry{
	left=30mm,
	top=20mm,
	right=20mm,
	bottom=20mm
}

\usepackage{listings} %подлючения листинга для кода
\usepackage{xcolor}

\definecolor{codegreen}{rgb}{0,0.6,0}
\definecolor{codegray}{rgb}{0.3,0.3,0.3}
\definecolor{codepurple}{rgb}{0.58,0,0.82}
\definecolor{backcolour}{rgb}{0.95,0.95,0.92}

% Значения по умолчанию
\lstset{
	rulecolor=\color{black},
	commentstyle=\color{codegreen},
	keywordstyle=\color{magenta},
	numberstyle=\scriptsize\color{codegray},
	stringstyle=\color{codepurple},
	basicstyle= \ttfamily\footnotesize,
	breakatwhitespace=true,% разрыв строк только на whitespacce
	breaklines=true,       % переносить длинные строки
	captionpos=b,          % подписи снизу -- вроде не надо
	numbers=left,          % нумерация слева
	numberstyle=\footnotesize,
	showspaces=false,      % показывать пробелы подчеркиваниями -- идиотизм 70-х годов
	showstringspaces=false,
	showtabs=false,        % и табы тоже
	stepnumber=1,
	tabsize=2,              % кому нужны табы по 8 символов?
	frame=single,
	stepnumber=1,
	extendedchars=\true,
	%numbersep=-6pt
}

% Стиль для псевдокода: строчки обычно короткие, поэтому размер шрифта побольше
\lstdefinestyle{pseudocode}{
	basicstyle=\small,
	keywordstyle=\color{black}\bfseries\underbar,
	language=Pseudocode,
	numberstyle=\footnotesize,
	commentstyle=\footnotesize\it
}

% Стиль для обычного кода: маленький шрифт
\lstdefinestyle{realcode}{
	basicstyle=\scriptsize,
	numberstyle=\footnotesize
}

% Стиль для коротких кусков обычного кода: средний шрифт
\lstdefinestyle{simplecode}{
	basicstyle=\footnotesize,
	numberstyle=\footnotesize
}

\renewcommand\lstlistingname{Листинг}
\renewcommand\lstlistlistingname{Листинги}


\usepackage{blindtext}

% Добавляем гипертекстовое оглавление в PDF
\usepackage{amssymb}
\usepackage{hyperref}
\hypersetup{
	colorlinks=true,
	unicode=true,
	urlcolor=black,
	linkcolor=black,
	anchorcolor=black,
	citecolor=black,
	menucolor=black,
	filecolor=black
}
\urlstyle{same}

\begin{document}
	\fefupage{Середа С.}{Б9119-01.03.02систпро}{}{т}{Вычислительная математика}
	\tableofcontents
	\pagebreak
	\section{Введение}
	Объектом исследования являются численные методы решения задач математической физики, а также программное обеспечение, реализующее эти методы, а именно: явные методы Адамса-Бэшфорта ---  ${k}$-го порядка погрешности решения задачи Коши для обыкновенного дифференциального уравнения 1-го порядка.
	
	Цель работы --- ознакомиться с численными методами анализа и решения обыкновенных дифференциальных уравнений, решить предложенные типовые задачи, сформулировать выводы по полученным решениям, отметить достоинства и недостатки методов, приобрести практические навыки и компетенции, а также опыт самостоятельной профессиональной деятельности, а именно:
	\begin{itemize}
		\item создать алгоритм решения поставленной задачи и реализовать его, протестировать программы;
		\item освоить теорию вычислительного эксперимента; современных компьютерных технологий; 
		\item приобрести навыки представления итогов проделанной работы в виде отчета, оформленного в соответствии с имеющимися требованиями, с привлечением современных средств редактирования и печати.
	\end{itemize}
	Работа над курсовым проектом предполагает выполнение следующих задач:
	\begin{itemize}
		\item дальнейшее углубление теоретических знаний обучающихся и их систематизацию;
		\item получение и развитие прикладных умений и практических навыков по направлению подготовки; 
		\item овладение методикой решения конкретных задач;
		\item развитие навыков самостоятельной работы;
		\item овладение методикой решения конкретных задач;
		\item развитие навыков обработки полученных результатов, анализа и осмысления их с учетом имеющихся литературных данных;
		\item приобретение навыков оформления описаний программного продукта;
		\item повышение общей и профессиональной эрудиции.
	\end{itemize}
	\section{Основная часть}
	\subsection{Постановка задачи}
	Требуется решить 5 задач на охлаждение и нагревание явным методом Адамса-Бэшфорта 3-го, 4-го, 5-го порядков погрешности решения задачи Коши для обыкновенного дифференциального уравнения 1-го порядка.
	\subsection{Описание алгоритмов решения задач}
	Методы Адамса ${k}$-го порядка требуют предварительного вычисления решения в ${k}$ начальных точках. Для вычисления начальных значений обычно используют одношаговые методы, например, 4-стадийный метод Рунге — Кутты 4-го порядка точности, которым мы и воспользовались.
	
	Локальная погрешность методов Адамса ${k}$-го порядка — ${O(h^{k})}$. Структура погрешности метода Адамса такова, что погрешность остается ограниченной или растет очень медленно в случае асимптотически устойчивых решений уравнения. Это позволяет использовать этот метод для отыскания устойчивых периодических решений, в частности, для расчета движения небесных тел.
	
	Для явного метода формулы будут:
	\begin{equation}
		\begin{aligned}
			y_{n+3} = y_{n+2} + h\left( \frac{23}{12} f(t_{n+2}, y_{n+2}) - \frac{4}{3} f(t_{n+1}, y_{n+1}) + \frac{5}{12}f(t_n, y_n)\right) , \\
		\end{aligned}
	\end{equation}
	\begin{equation}
		\begin{aligned}
			y_{n+4} = y_{n+3} + h&\left( \frac{55}{24} f(t_{n+3}, y_{n+3}) - \frac{59}{24} f(t_{n+2}, y_{n+2}) +\right. \\ &\left. \frac{37}{24} f(t_{n+1}, y_{n+1}) - \frac{3}{8} f(t_n, y_n) \right) , \\
		\end{aligned}
	\end{equation}
	\begin{equation}
		\begin{aligned}
			y_{n+5} = y_{n+4} + h&\left( \frac{1901}{720} f(t_{n+4}, y_{n+4}) - \frac{1387}{360} f(t_{n+3}, y_{n+3}) +\right. \\ &\left. \frac{109}{30} f(t_{n+2}, y_{n+2}) - \frac{637}{360} f(t_{n+1}, y_{n+1}) + \frac{251}{720} f(t_n, y_n) \right) .
		\end{aligned}
	\end{equation}
	Классический метод Рунге — Кутты четвёртого порядка:
	
	Рассмотрим задачу Коши для системы обыкновенных дифференциальных уравнений первого порядка. (Далее ${\mathbf{y},\mathbf{f},\mathbf{k}_{i}\in\mathbb{R}^{n}}$, а ${x,h\in\mathbb{R}^{1}}).$
	$$y'=f(x,y),\;y(x_0)=y_0$$
	Тогда приближенное значение в последующих точках вычисляется по итерационной формуле:
	$$y_{n+1} = y_n + {h \over 6}(k_1 + 2k_2 + 2k_3 + k_4)$$
	Вычисление нового значения проходит в четыре стадии:
	\begin{align}
		k_1&=f(x_n,y_n),\\
		k_2&=f(x_n+\dfrac{h}{2},y_n+\dfrac{h}{2}k_1),\\
		k_3&=f(x_n+\dfrac{h}{2},y_n+\dfrac{h}{2}k_2),\\
		k_4&=f(x_n+h,y_n+hk_3).
	\end{align}
	где $h$ --- величина шага сетки по $x$.
	
	Этот метод имеет четвертый порядок точности. Это значит, что ошибка на одном шаге имеет порядок $O(h^5)$, а суммарная ошибка на конечном интервале интегрирования имеет порядок $O(h^4)$ .
	
	Таким образом, мы сначала высчитываем с помощью метода Рунге — Кутты необходимый шаг для формулы Адамса-Бэшфорта, а затем уже его вычисляли.
	\subsection{Описание тестов, использованных для отладки}
	\subsection{Вычислительные эксперименты}
		\subsubsection{Задача 1}
			\subsubsection*{Постановка задачи}
				Тело охладилось за 10 мин от 70 до 40 °C. Температура окружающей среды поддерживается равной 25 °C. Сколько еще минут понадобится, чтобы тело остыло до 30 °C?
			\subsubsection*{Решение}
			 
			\subsubsection*{Графическое отображение}
				\mypicture{24}{24 задачи}
		\subsubsection{Задача 2}
			\subsubsection*{Постановка задачи}
				Килограмм только что размороженной воды,помещенной в сосуд с хорошей теплоизоляцией, нагревается спиралью, напряжение на которую подается равномерно и к концу десятой минуты достигает 120 В. До какой температуры нагреется вода к этому моменту времени,если при 20 °C сопротивление спирали составляет 20 Ом, а ее температурный коэффициент сопротивления равен 0,004 °$\text{C}^{-1}$.
			\subsubsection*{Решение}
			\subsubsection*{Графическое отображение}
				\mypicture{25}{25 задачи}
		\subsubsection{Задача 3}
			\subsubsection*{Постановка задачи}
				Найти атмосферное давление на высоте 500 м над уровнем моря, пренебрегая изменениями температуры воздуха на этой высоте.
			\subsubsection*{Решение}
			\subsubsection*{Графическое отображение}
				\mypicture{26}{26 задачи}
		\subsubsection{Задача 4}
			\subsubsection*{Постановка задачи}
				За какое время вытечет вся вода из цилиндрического бака диаметром 2 м и высотой 5 м, поставленного вертикально, через отверстие в его дне площадью 2 см2? Как изменится ответ, если бак расположить горизонтально и отверстие просверлить в самой нижней части боковой поверхности бака?
			\subsubsection*{Решение}
			\subsubsection*{Графическое отображение}
				\mypicture{27_1}{27 задачи}
		\subsubsection{Задача 5}
			\subsubsection*{Постановка задачи}
				Воронка имеет форму конуса радиусом 9 см и высотой 25 см, обращенного вершиной вниз. За какое время вся вода вытечет из воронки через круглое отверстие со спрямленными краями диаметром 6 мм, расположенное в вершине конуса?
			\subsubsection*{Решение}
			\subsubsection*{Графическое отображение}
				\mypicture{27_2}{27 задачи}
	\subsection{Трудности и спорные вопросы, которые возникли по конкретным 
		видам работы}
		\begin{enumerate}
			\item Одной из проблем, которая вводила в заблуждение и ставила
			под сомнение правильность алгоритма – одинаковые результаты вычислений методом Адамса-Бэшфорта 3-го, 4-го, 5-го порядков погрешности для решения задач;
		\end{enumerate}
	\subsection{Пути их разрешения}
	\section{Заключение}
	\subsection{Полученные результаты и их анализ}
	В результате работы над курсовым проектом приобрел практические навыки владения:
	\begin{itemize}
		\item современными численными методами решения задач математической физики;
		\item основами алгоритмизации для численного решения задач математической физики на одном из языков программирования;
		\item инструментальными средствами, поддерживающими разработку программного обеспечения для численного решения задач математической физики;
	\end{itemize}
	а также навыками представления итогов проделанной работы в виде отчета, оформленного в соответствии с имеющимися требованиями, с привлечением современных средств редактирования и печати.
	\pagebreak
	\section{Список использованных источников}
		\begin{enumerate}
			\item Гриншпон, Я.С.  Геометрические, физические и экономическиезадачи, сводящиеся к дифференциальным уравнениям: учебное пособие. --- Томск : Издательство Томского государственного университетасистем управления и радиоэлектроники. --- 2011. --- 74 c.
			\item Википедия[Электронный ресурс]. – \\
			Режим доступа: https://ru.wikipedia.org/wiki/МетодАдамса. – Дата доступа: 29.12.2021.
		\end{enumerate}
		\pagebreak
	\section{Приложения}
	\begin{lstlisting}[language=Python, caption=Файл <<methods.py>> с реализацией методов Рунге-Кутты и Адамса-Бэшфорта]
		import math
		
		# Рунге-Кутты 4 порядка
		# f - дифференциальное уравнение, y - функция, x - переменная, h - шаг, N - количество
		# возвращает массив, вычисленных y и x, количеством N
		def RK4_step(f, y, x, h, N):
		y_mass = [y]
		x_mass = [x]
		for k in range(N):
		k1 = f(y, x)
		k2 = f(y + k1 / 2, x + h / 2)
		k3 = f(y + k2 / 2, x + h / 2)
		k4 = f(y + k3, x + h)
		y = y + (k1 + 2 * k2 + 2 * k3 + k4) * h / 6
		x = x + h
		y_mass.append(y)
		x_mass.append(x)
		return y_mass, x_mass
		
		# Метод Адамса-Бэшфорта 3-го порядка
		def AB_3_step(mass_y, mass_x, f, h):
		return mass_y[-1] + h * (23 / 12 * f(mass_y[-1], mass_x[-1]) -
		4 / 3 * f(mass_y[-2], mass_x[-2]) +
		5 / 12 * f(mass_y[-3], mass_x[-3]))
		
		# Метод Адамса-Бэшфорта 4-го порядка
		def AB_4_step(mass_y, mass_x, f, h):
		return mass_y[-1] + h * (55 / 24 * f(mass_y[-1], mass_x[-1]) -
		59 / 24 * f(mass_y[-2], mass_x[-2]) +
		37 / 24 * f(mass_y[-3], mass_x[-3]) -
		3 / 8 * f(mass_y[-4], mass_x[-4]))
		
		# Метод Адамса-Бэшфорта 5-го порядка
		def AB_5_step(mass_y, mass_x, f, h):
		return mass_y[-1] + h * (1901 / 720 * f(mass_y[-1], mass_x[-1]) -
		1387 / 360 * f(mass_y[-2], mass_x[-2]) +
		109 / 30 * f(mass_y[-3], mass_x[-3]) -
		637 / 360 * f(mass_y[-4], mass_x[-4]) +
		251 / 720 * f(mass_y[-5], mass_x[-5]))
		
		# Дифф. уравнение температуры от времени 24-й задачи
		def dif_24(T, t):
		k = math.log(3) / 10
		return -k * (T - 25)
		
		# Дифф. уравнение температуры от времени 25-й задачи
		def dif_25(T, t):
		return t ** 2 / (2100000 * (0.92 + 0.004 * T))
		
		# Дифф. уравнение давления от высоты 26-й задачи
		def dif_26(p, h):
		k = 0.000012
		g = 9.8
		return -k * p * g
		
		# Дифф. уравнение высоты воды от времени 27-й задачи
		def dif_27_1(h, t):
		return -0.000248 * (5 * h) ** (1 / 2) / math.pi
		
		def dif_27_2(h, t):
		return -0.0000248 * (5 / (2 - h)) ** (1 / 2)
		
		# Дифф. уравнение высоты воды от времени 28-й задачи
		def dif_28(h, t):
		return -1.53 * 10 ** (-5) * 5 ** (1 / 2) * h ** (-3 / 2) / 0.1296
	\end{lstlisting}
	\newpage
	\begin{lstlisting}[language=Python, caption=Файл <<main.py>> с решением дифференциальных уравнений для наших задач]
		from methods import *
		eps = 0.001
		def task_24():
		T, t = RK4_step(dif_24, 70, 0, eps, 2)
		while T[-1] >= 30:
		T.append(AB_3_step(T, t, dif_24, eps))
		t.append(t[-1] + eps)
		result = t[-1]
		print('result 24 task =', result - 10)
		
		def task_25():
		T, t = RK4_step(dif_25, 0, 0, eps, 3)
		while t[-1] <= 600:
		T.append(AB_4_step(T, t, dif_25, eps))
		t.append(t[-1] + eps)
		result = T[-1]
		print('result 25 task =', result)
		
		def task_26():
		p, h = RK4_step(dif_26, 101325, 0, eps, 4)
		while h[-1] <= 500:
		p.append(AB_5_step(p, h, dif_26, eps))
		h.append(h[-1] + eps)
		result = p[-1]
		print('result 26 task =', result)
		
		def task_27():
		h, t = RK4_step(dif_27_1, 5, 0, eps, 4)
		while h[-1] >= 0:
		h.append(AB_5_step(h, t, dif_27_1, eps))
		t.append(t[-1] + eps)
		result = t[-1]
		print('first result 27 task =', result)
		h, t = RK4_step(dif_27_2, 1.999999999, 0, eps, 2)
		while h[-1] >= 0:
		h.append(AB_3_step(h, t, dif_27_2, eps))
		t.append(t[-1] + eps)
		result = t[-1]
		print('second result 27 task =', result)
		
		def task_28():
		h, t = RK4_step(dif_28, 0.25, 0, eps, 3)
		while h[-1] > 0:
		h.append(AB_4_step(h, t, dif_28, eps))
		t.append(t[-1] + eps)
		result = t[-1]
		print('result 28 task =', result)
		
		if __name__ == '__main__':
		task_24(), task_25(), task_26(), task_27(), task_28()
	\end{lstlisting}
\end{document}